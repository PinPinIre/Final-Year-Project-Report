\chapter{Introduction}

\section{Introduction}

In this chapter I will give the motivation and the reasoning for the research carried out.
Following this, the research question is stated, the key objectives and challenges are examined and finally the outline of the report structure is laid out.

\section{Motivation}

Since ancient times, humankind has collated and archived the written knowledge of our species into collections.
From the Library of Alexandria to the modern day public library, humanity seems to have a need to preserve the knowledge of one generation to be presented to the next.
These collections have proved invaluable for students and scholars wishing to expand their own knowledge and understanding.

In recent years there has been a explosion in online learning resources.
From projects like the Internet Archive, Wikipedia, Project Gutenberg, and the rapid rise of the MOOC (Massive Open Online Course), we can see the human need to preserve and share knowledge continuing.
These systems, however, only provide the potential student with access to resources, they do not guarantee quality nor do they always facilitate a logical pathway to knowledge or understanding.

In the pre-digital age, some of the knowledge in these documents would have been catalogued by a human to make access easier.
The Dewey Decimal system used by libraries is an excellent example of the manual cataloguing of books to set topics.
However in the current digital age the flood of information is too huge to be finely catalogued by human operators.
There is also the problem of cataloging documents which could potentially contain a large number of varying topics.
This presents learners with the challenge of finding relevant information to suit their needs.

Often learners may be looking for information related to something they have already read, but are unable to bridge the gap between what they have already read, and what they should read next.
The aim of this project is to evaluate the use of popular machine learning algorithms to generate recommendations for a student.

More precisely this project will evaluate the performance of Latent Dirichlet Allocation, K-Nearest Neighbors, and Word2Vec.
The performance will be evaluated on temporal performance, quality of results and results compared to a gold standard.
Furthermore, this project will detail the background of each of the algorithms.

\section{Research Question}
The main research question that this project hopes to answer is; can we identify the best algorithm for resource recommendations by comparing their output on a varying corpus of academic papers?

\section{Objectives}
In order to answer the above research question there are three objectives we must first complete.

\begin{enumerate}
    \item Appropriate algorithms for generating recommendations must be first researched.
    From the algorithms researched, the most relevant today for generating recommendations must be chosen.

    \item A corpus must be selected that is relatively large and must contain a variety of topics.
    The corpus should also contain educational documents.

    \item The algorithms being evaluated are unsupervised and thus have no ideal model to compare against.
    To identify the best algorithm to use, an appropriate performance criteria must be decided on.

    \item Once the above objectives have been completed the algorithms chosen must be applied to the selected corpus.
    An evaluation of the performance of each of the algorithms must then be conducted.
\end{enumerate}

\section{Challenges}
As well as the objectives outlined above there are a number of challenges that must overcome before the research question can be answered.

\begin{enumerate}
    \item The selection of an appropriate educational corpus poses a problem due to the closed nature of publishing.
    Access to many academic papers and journals are often restricted to subscribers or members of professional organisations.

    \item The corpus selected must be a respected academic resource.
    As the research project is investigating the generation of recommendations in an educational context, the quality of the corpus should be to a high standard.

    \item Any algorithms or libraries chosen for the project will have differences which will have to be overcome.
    These differences could vary from function API's to how data is represented internally.

    \item The algorithms being investigated are unsupervised.
    This means that there is not a gold standard or previous model to compare the results to.
    The evaluation of the generated recommendations is therefore difficult, as it is largely subjective.
\end{enumerate}

\section{Thesis Outline}
Chapter 2 gives an analysis of machine learning and a detailed account of the current state of the art in content filtering.
This chapter will finish with a review of previous research in this field and how this project relates to that work.

Chapter 3 will describe the approach taken towards the problems, and the various design decisions that were made throughout the project.
This chapter will contain a breakdown of the project design as well as diagrams of the system architecture.

Chapter 4 will be an evaluation of the system design presented in Chapter 3.
Best practices and challenges encountered will be discussed in this chapter.
An unbiased evaluation of the system will also be made.

Chapter 5 will be an evaluation of the experiment results and a measure of the system's performance.
The criteria for a good learning resource will also be presented as well as any non-functional requirements.
A discussion of each of the algorithms chosen will also be presented individually and as a group.

Chapter 6 will conclude the report with a review of the project and results obtained.
A short overview of future work and how the system designed could be improved will also be presented.
