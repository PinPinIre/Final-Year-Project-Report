\chapter{Conclusion}
In this chapter there will be discussion of what this project achieved and how it addressed the research question as stated in chapter 1.
Any future work which could be done based on the work of this project will also be discussed.

\section{Summary}
The overall aim of this research project was to evaluate the performance of machine learning algorithms and pick the best one for generating document recommendations.

In chapter 1 the main objectives and challenges were outlined and briefly discussed.
During the course of this project: the state-of-the-art algorithms which could have been used were researched, a large corpus of educational documents was sourced, the corpus was processed into the appropriate formats, models for each of the algorithms being evaluated were built, and an evaluation of the performance criteria as outlined in chapter 3 was conducted.

The temporal performance and quality of recommendations for LDA, k-NN and Word2Vec were evaluated and ranked on their overall performance.
It would seem that LDA was the best of the three algorithms for the corpus used.
LDA performed well for temporal performance and it consistently generated the best recommendations.
From this it can be concluded that of the 3 algorithms investigated, LDA is the better of the three if the intention is to generate document similarity recommendations.

\section{Future Work}
Further work should be carried out on this project to evaluate the performance of the machine learning algorithms.
Unfortunately only three algorithms were investigated but there are other machine learning algorithms which could have been used.
LDA is heavily influenced by LSI and pLSI and it would have been interesting to directly compare the performance of these algorithms alongside the ones investigated.

Another interesting piece of future work could be done to investigate the use of LDA and Word2Vec to summarise documents.
When presented with a document, LDA can be used to find the topic which has the highest influence on it.
As LDA's topics are constructed of word probabilities, they can be used to find words related to a document.
Word2Vec on the other hand can be used to find the words most similar to a given document.
Further work could be done to investigate the performance of LDA and Word2Vec for this type of categorisation.
